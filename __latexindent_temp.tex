\documentclass[cn,10pt,math=newtx,chinesefont=founder]{elegantbook}

\title{台北復興暑期加課}
\subtitle{2021年}

\author{李宥頡}
\institute{National Taiwan University}
%\date{May 2, 2021}
%\version{4.1}
%\bioinfo{自定义}{信息}


\setcounter{tocdepth}{3}

%\logo{logo-blue.png}
\cover{cover.jpg}

% 本文档命令
\usepackage{array}
\newcommand{\ccr}[1]{\makecell{{\color{#1}\rule{1cm}{1cm}}}}

\definecolor{customcolor}{RGB}{32,178,170}
\colorlet{coverlinecolor}{customcolor}



\begin{document}
\maketitle
\frontmatter
\mainmatter
\chapter{波}
\section{波的性質}
\subsection{定義}
\begin{enumerate}
    \item 波:物質受力後產生上下起伏或疏密相間的現象
    \item 波動:物體受外力擾動,引發鄰近物質被擾動,稱為波動;一開始被擾動的地方稱為波源
\end{enumerate}
\subsection{特性}


\begin{enumerate}
    \item 物質振動時,具有能量,故波是一種能量的展現
    \item 波只傳遞能量,不傳遞介質
    \begin{note}{結論?}
    \end{note}
\end{enumerate}

\subsection{波的分類}
\vspace{25ex}
\subsection{名詞介紹}
\begin{enumerate}
    \item 頻率$f$:質點一秒振動的次數,以Hz赫茲作為單位
    \item 週期$T$:經過一次完整的波需要的時間
    \begin{note}{週期和頻率的關係?}
        \\
    \end{note}
    \item 波峰:介質振動最高處,加速度最大
    \item 波谷:介質振動最低處,加速度最大
    \item 平衡位置:還沒振動處,速度最大
    \item 波長$\lambda$:
    \begin{enumerate}
        \item 兩相鄰波峰
        \item 兩相鄰波谷
        \item 其他?
    \end{enumerate}
    \item 振幅$R$:平衡點到波峰(谷)的距離
    \begin{note}{一個週期經過幾次振幅?}
        \\
    \end{note}
    \item 波速$v$:\\
    
\end{enumerate}
\newpage

\subsection{結論}
\begin{enumerate}
    \item 波速只和介質有關
    \item 頻率由波源決定
    \item 波長為被決定的物理量
\end{enumerate}

\section{波的種類}
\subsection{繩波}
\begin{enumerate}
    \item 質點振動方向和波前進方向垂直$\longrightarrow$ 橫波
    \item 需要介質傳播$\longrightarrow$ 力學波
    \item 波速
    \begin{enumerate}
        \item $v=f\lambda$
        \item $v=\sqrt{\frac{F}{\mu}}$\\
    \end{enumerate}
    \item 波速與頻率\_\_\_\_\_\_;與波長\_\_\_\_\_\_;與振幅\_\_\_\_\_\_;與介質\_\_\_\_\_\_
\end{enumerate}

\subsection{聲波}
\begin{enumerate}
    \item 氣體分子振動方向與波前進方向平行$\longrightarrow$ 縱波
    \item 需要介質傳播$\longrightarrow$ 力學波
    \begin{note}{太空有聲音嗎?}
        
    \end{note}
    \item 波速 $v=331+0.6t, t$為攝氏溫度,只和介質有關
    \item 聲音三要素
    \begin{table}[htbp]

  
      \begin{tabular}{llll}
      \toprule
      要素 & 意義 & 物理量 & 單位 \\
      \midrule
      響度(音量) & 大小聲 &    &  \\
      音調 & 高低音 &    &  \\
      音色 & 特性 &    &   \\

      \bottomrule
      \end{tabular}%

    \end{table}%
\end{enumerate}



\chapter{光}


\end{document}