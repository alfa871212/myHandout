\documentclass[cn,10pt,math=newtx,chinesefont=founder]{../elegantbook}

\title{半導體物理}
\subtitle{2021年}

\author{李宥頡}
\institute{National Taiwan University}
%\date{May 2, 2021}
%\version{4.1}
%\bioinfo{自定义}{信息}


\setcounter{tocdepth}{3}

%\logo{logo-blue.png}
\cover{cover.jpg}

% 本文档命令
\usepackage{array}
\newcommand{\ccr}[1]{\makecell{{\color{#1}\rule{1cm}{1cm}}}}

\definecolor{customcolor}{RGB}{32,178,170}
\colorlet{coverlinecolor}{customcolor}

\begin{document}

\maketitle
\frontmatter

\chapter*{序}
\tableofcontents

\mainmatter

\chapter{金屬理論}
\section{Drude Theory}
\subsection{基本假設}
%Ref from A&M chap 1 
相較於氣體動力論中,只有一種粒子,在金屬中至少有兩種粒子,一為帶負電的電子,另一為帶正電的金屬離子(因為金屬是電中性),
在Drude model裡,金屬可視為不可移動,且質量比電子重很多的粒子。若原子的原子序為$Z_a$,即代表中性原子具有$Z_a$個電子,
其中有$Z$個電子是相對來說束縛較鬆的,所以金屬將會丟出這些電子,並參與化學反應,稱為傳導電子(conduction electron),
留下的$Z_a-Z$個電子相對靠近原子核,並對化學反應貢獻較少,稱為core elctron。\\
以下為Drude Theory的假設
\begin{enumerate}
    \item 電子在運動碰撞時,忽略和其他電子的交互作用(independent electorn approximation),也忽略
          和其他離子的交互作用(free electron approximation)
    \begin{enumerate}
        \item  若無外加電磁場在金屬上,電子在碰撞之間會作等速度運動
        \item  若有外加電磁場在金屬上,電子在碰撞之間會作等加速度運動
    \end{enumerate}    

\end{enumerate}


\chapter{能帶}
\section{}

原子的電子在原子核的勢場與其他電子的作用下,會分別位在不同的能階,形成所謂的電子殼層。透過三個量子數($n, l ,m$),可描述對應的電子
軌域,例如($2p$),每一殼層對應不同的能量。但在原子相互接近時,不同原子的電子雲會互相重疊。



\chapter{Electron and Holes in Semiconductors \\ (Chenming Chap1)}

\section{Si wafers and crystal planes}

For 4 inch Si wafer, P for 90 degree N for 180 degree
 
\section{Bond model}

\end{document}