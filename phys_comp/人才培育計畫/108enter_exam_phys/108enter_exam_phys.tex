\documentclass[10pt]{article}
\usepackage[utf8]{inputenc}
\usepackage[T1]{fontenc}
\usepackage{CJKutf8}
\usepackage{amsmath}
\usepackage{amsfonts}
\usepackage{amssymb}
\usepackage{mhchem}
\usepackage{stmaryrd}

\begin{document}
\begin{CJK}{UTF8}{mj}國立臺灣大學\end{CJK} 108 \begin{CJK}{UTF8}{mj}學年度高中物理科學人才培育計畫\end{CJK} \begin{CJK}{UTF8}{mj}物理科試題\end{CJK} (108 \begin{CJK}{UTF8}{mj}新生\end{CJK})

(\begin{CJK}{UTF8}{mj}重力加速度之值以\end{CJK} 10.0 \begin{CJK}{UTF8}{mj}公尺\end{CJK}/\begin{CJK}{UTF8}{mj}秒\end{CJK} ${ }^{2}$ \begin{CJK}{UTF8}{mj}計算\end{CJK})

\begin{enumerate}
  \item \begin{CJK}{UTF8}{mj}有一人騎著自行車由靜止以\end{CJK} $0.10$ \begin{CJK}{UTF8}{mj}公尺\end{CJK}/\begin{CJK}{UTF8}{mj}秒\end{CJK} $^{2}$ \begin{CJK}{UTF8}{mj}之等加速度加速前進\end{CJK} 60 \begin{CJK}{UTF8}{mj}秒後就維\end{CJK} \begin{CJK}{UTF8}{mj}持等速前進\end{CJK},\begin{CJK}{UTF8}{mj}再騎了\end{CJK} 5 \begin{CJK}{UTF8}{mj}分鐘後\end{CJK},\begin{CJK}{UTF8}{mj}他便以\end{CJK}-0.05 \begin{CJK}{UTF8}{mj}公尺\end{CJK}/\begin{CJK}{UTF8}{mj}秒\end{CJK} $^{2}$ \begin{CJK}{UTF8}{mj}之等加速度行進\end{CJK},\begin{CJK}{UTF8}{mj}直\end{CJK} \begin{CJK}{UTF8}{mj}到車停下來\end{CJK}。\begin{CJK}{UTF8}{mj}請問\end{CJK}(a)\begin{CJK}{UTF8}{mj}他共騎了多少時間\end{CJK}(1)(b)\begin{CJK}{UTF8}{mj}他共騎了多少公尺\end{CJK}(2) 。

  \item \begin{CJK}{UTF8}{mj}一質量為\end{CJK} $0.010$ \begin{CJK}{UTF8}{mj}公斤\end{CJK}、\begin{CJK}{UTF8}{mj}動能為\end{CJK} 50 \begin{CJK}{UTF8}{mj}焦耳之子彈射向一置於光滑地面上之靜止木\end{CJK} \begin{CJK}{UTF8}{mj}塊後停留在該木塊內\end{CJK} 。\begin{CJK}{UTF8}{mj}若該木塊之質量為\end{CJK} $1.99$ \begin{CJK}{UTF8}{mj}公斤\end{CJK} , \begin{CJK}{UTF8}{mj}則在子彈射入木塊後\end{CJK} , \begin{CJK}{UTF8}{mj}木塊\end{CJK}-\begin{CJK}{UTF8}{mj}子彈系統之速率為\end{CJK} (3) \begin{CJK}{UTF8}{mj}公尺\end{CJK}/\begin{CJK}{UTF8}{mj}秒\end{CJK}。

  \item \begin{CJK}{UTF8}{mj}一質量為\end{CJK} 1000 \begin{CJK}{UTF8}{mj}公斤\end{CJK} ,\begin{CJK}{UTF8}{mj}初速為\end{CJK} $20.0$ \begin{CJK}{UTF8}{mj}公尺\end{CJK}/\begin{CJK}{UTF8}{mj}秒之汽車受到一定值之摩擦力而開始\end{CJK} \begin{CJK}{UTF8}{mj}減速\end{CJK} , \begin{CJK}{UTF8}{mj}經過\end{CJK} 100 \begin{CJK}{UTF8}{mj}公尺後該車完全停止\end{CJK}。\begin{CJK}{UTF8}{mj}該車所受到之摩擦力為\end{CJK} (4) \begin{CJK}{UTF8}{mj}牛頓\end{CJK}。 \begin{CJK}{UTF8}{mj}如果此摩擦力完全來自地面與輪胎間之摩擦力\end{CJK},\begin{CJK}{UTF8}{mj}則地面與輪胎間之動摩擦係\end{CJK} \begin{CJK}{UTF8}{mj}數為\end{CJK} (5) $^{\circ}$

  \item \begin{CJK}{UTF8}{mj}一質量為\end{CJK} $2.0$ \begin{CJK}{UTF8}{mj}公斤\end{CJK} , \begin{CJK}{UTF8}{mj}速率為\end{CJK} 6 \begin{CJK}{UTF8}{mj}公尺\end{CJK}/\begin{CJK}{UTF8}{mj}秒之物體與另一靜止之物體做完全彈性碰\end{CJK} \begin{CJK}{UTF8}{mj}撞後\end{CJK},\begin{CJK}{UTF8}{mj}速率變為原來速率之\end{CJK} 3 \begin{CJK}{UTF8}{mj}分之\end{CJK} 1 , \begin{CJK}{UTF8}{mj}繼續向原來的方向前進\end{CJK}。\begin{CJK}{UTF8}{mj}請問該靜止\end{CJK} \begin{CJK}{UTF8}{mj}之物體的質量為\end{CJK} (6) \begin{CJK}{UTF8}{mj}公斤\end{CJK} 。\begin{CJK}{UTF8}{mj}碰撞後該原先靜止之物體的動能變為\end{CJK} (7) \begin{CJK}{UTF8}{mj}焦耳\end{CJK}

  \item \begin{CJK}{UTF8}{mj}物體在運動時所受到空氣的阻力的大小可表示為\end{CJK} $F_{\mathrm{D}}=\mathrm{Dv}^{2}$, \begin{CJK}{UTF8}{mj}其中\end{CJK} $\mathrm{D}$ \begin{CJK}{UTF8}{mj}為一常數\end{CJK} (\begin{CJK}{UTF8}{mj}和物體的截面積以及空氣的密度有關\end{CJK})。\begin{CJK}{UTF8}{mj}今有甲乙兩顆體積相同之圓球形物\end{CJK} \begin{CJK}{UTF8}{mj}體從高空落下\end{CJK},\begin{CJK}{UTF8}{mj}在落地之前皆達到終端速度\end{CJK} 。\begin{CJK}{UTF8}{mj}甲的質量是乙的質量之\end{CJK} 4 \begin{CJK}{UTF8}{mj}倍\end{CJK} \begin{CJK}{UTF8}{mj}在著地之瞬間\end{CJK},\begin{CJK}{UTF8}{mj}物體甲之速率為物體乙速率之\end{CJK}(8) \begin{CJK}{UTF8}{mj}倍\end{CJK}。

  \item \begin{CJK}{UTF8}{mj}一行星的質量為地球之兩倍\end{CJK}, \begin{CJK}{UTF8}{mj}半徑為地球之\end{CJK} $1.5$ \begin{CJK}{UTF8}{mj}倍\end{CJK}。\begin{CJK}{UTF8}{mj}在該行星表面的重力加\end{CJK} \begin{CJK}{UTF8}{mj}速度\end{CJK} $\mathrm{g}$ \begin{CJK}{UTF8}{mj}之值為\end{CJK} (9) \begin{CJK}{UTF8}{mj}公尺\end{CJK}/\begin{CJK}{UTF8}{mj}秒\end{CJK} $^{2}$ 。

  \item \begin{CJK}{UTF8}{mj}一繩子的一端綁著一顆質量為\end{CJK} $m$ \begin{CJK}{UTF8}{mj}的石頭並將該石頭在垂直方向作圓周運動\end{CJK} (\begin{CJK}{UTF8}{mj}半徑為\end{CJK} $R)$ 。\begin{CJK}{UTF8}{mj}如果當石頭在最低點時\end{CJK},\begin{CJK}{UTF8}{mj}繩子所受到的張力是石頭重量的\end{CJK} 5 \begin{CJK}{UTF8}{mj}倍\end{CJK}。 \begin{CJK}{UTF8}{mj}則石頭在該點的速率為\end{CJK} (10) , \begin{CJK}{UTF8}{mj}當石頭在最高為點的速率為\end{CJK} (11) (\begin{CJK}{UTF8}{mj}繩\end{CJK} \begin{CJK}{UTF8}{mj}子之質量可以忽略\end{CJK})

  \item \begin{CJK}{UTF8}{mj}高速公路上有一救護車邊行駛邊發出\end{CJK} 300 \begin{CJK}{UTF8}{mj}赫茲的警笛聲\end{CJK}。\begin{CJK}{UTF8}{mj}救護車的時速為\end{CJK} 108 \begin{CJK}{UTF8}{mj}公里\end{CJK}, \begin{CJK}{UTF8}{mj}請寫出在救護車正前方\end{CJK}(12) \begin{CJK}{UTF8}{mj}和正後方\end{CJK}(13) \begin{CJK}{UTF8}{mj}測到該警笛聲的波長之\end{CJK} \begin{CJK}{UTF8}{mj}值\end{CJK}。(\begin{CJK}{UTF8}{mj}聲波在空氣中的傳遞速率為\end{CJK} 330 \begin{CJK}{UTF8}{mj}公尺\end{CJK}/\begin{CJK}{UTF8}{mj}秒\end{CJK})

  \item \begin{CJK}{UTF8}{mj}有一實心圓柱\end{CJK},\begin{CJK}{UTF8}{mj}其長度為其半徑之\end{CJK} 12 \begin{CJK}{UTF8}{mj}倍\end{CJK}。\begin{CJK}{UTF8}{mj}若欲在相同溫度下使其幅射熱變成\end{CJK} \begin{CJK}{UTF8}{mj}原來之兩倍\end{CJK} , \begin{CJK}{UTF8}{mj}須將該圓柱切成\end{CJK}(14) \begin{CJK}{UTF8}{mj}塊相同長度之小圓柱\end{CJK}。

  \item \begin{CJK}{UTF8}{mj}以一加熱器加熱\end{CJK} 800 \begin{CJK}{UTF8}{mj}克的純水\end{CJK},\begin{CJK}{UTF8}{mj}升高\end{CJK} $20^{\circ} \mathrm{C}$ \begin{CJK}{UTF8}{mj}費時\end{CJK} 40 \begin{CJK}{UTF8}{mj}秒\end{CJK};\begin{CJK}{UTF8}{mj}而加熱\end{CJK} 400 \begin{CJK}{UTF8}{mj}克的某液\end{CJK} \begin{CJK}{UTF8}{mj}體時\end{CJK},\begin{CJK}{UTF8}{mj}升高\end{CJK} $10^{\circ} \mathrm{C}$ \begin{CJK}{UTF8}{mj}費時\end{CJK} 20 \begin{CJK}{UTF8}{mj}秒\end{CJK}, \begin{CJK}{UTF8}{mj}則該液體的比熱為\end{CJK}(15) \begin{CJK}{UTF8}{mj}卡\end{CJK}/\begin{CJK}{UTF8}{mj}克\end{CJK} ${ }^{\circ} \mathrm{C} ; 400$ \begin{CJK}{UTF8}{mj}克該液\end{CJK} \begin{CJK}{UTF8}{mj}體的熱容量為\end{CJK} (16) \begin{CJK}{UTF8}{mj}卡\end{CJK} $/{ }^{\circ} \mathrm{C}$

  \item \begin{CJK}{UTF8}{mj}以\end{CJK} 100 \begin{CJK}{UTF8}{mj}牛頓的力作用於一彈簧上會使該彈簧壓缩\end{CJK} $2.0$ \begin{CJK}{UTF8}{mj}公分\end{CJK} , \begin{CJK}{UTF8}{mj}若將該彈簧壓缩\end{CJK} $3.0$ \begin{CJK}{UTF8}{mj}公分\end{CJK} , \begin{CJK}{UTF8}{mj}此時彈簧之位能為\end{CJK} (17) \begin{CJK}{UTF8}{mj}焦耳\end{CJK}。

  \item \begin{CJK}{UTF8}{mj}已知在某地區接受到日光平均強度為\end{CJK} 100 \begin{CJK}{UTF8}{mj}瓦\end{CJK}/\begin{CJK}{UTF8}{mj}平方公尺\end{CJK} , \begin{CJK}{UTF8}{mj}平均每天日照時間\end{CJK} 6 \begin{CJK}{UTF8}{mj}小時\end{CJK},\begin{CJK}{UTF8}{mj}而太陽能板可將\end{CJK} $20 \%$ \begin{CJK}{UTF8}{mj}的太陽能轉換成電能\end{CJK}。\begin{CJK}{UTF8}{mj}若某戶住家平均每天耗電\end{CJK} \begin{CJK}{UTF8}{mj}為\end{CJK} 12 \begin{CJK}{UTF8}{mj}度電\end{CJK},\begin{CJK}{UTF8}{mj}則需裝設\end{CJK} (18) \begin{CJK}{UTF8}{mj}平方公尺的太陽能板\end{CJK},\begin{CJK}{UTF8}{mj}方可自給自足\end{CJK}。

  \item \begin{CJK}{UTF8}{mj}假設某人測得的收縮壓為\end{CJK} $1.33 \times 10^{4} \mathrm{~Pa}$ (\begin{CJK}{UTF8}{mj}約\end{CJK} 100 \begin{CJK}{UTF8}{mj}毫米汬柱\end{CJK}),\begin{CJK}{UTF8}{mj}請問在他的心臟上\end{CJK}

\end{enumerate}
\begin{CJK}{UTF8}{mj}方\end{CJK} 35 \begin{CJK}{UTF8}{mj}公分的頭部某一點的血壓為多少\end{CJK}(19)?\begin{CJK}{UTF8}{mj}在他腳底\end{CJK}(\begin{CJK}{UTF8}{mj}在心臟下方\end{CJK} 110 \begin{CJK}{UTF8}{mj}公分\end{CJK}) \begin{CJK}{UTF8}{mj}的血厭為多少\end{CJK} (20)? (\begin{CJK}{UTF8}{mj}血液的密度為\end{CJK} $1.06 \times 10^{3}$ \begin{CJK}{UTF8}{mj}公斤\end{CJK}/\begin{CJK}{UTF8}{mj}立方公尺\end{CJK})


\end{document}