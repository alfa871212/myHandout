\documentclass[cn,10pt,math=newtx,chinesefont=founder]{elegantbook}

\title{林口康橋作業}
\subtitle{2021年}

\author{李宥頡}
\institute{National Taiwan University}
%\date{May 2, 2021}
%\version{4.1}
%\bioinfo{自定义}{信息}


\setcounter{tocdepth}{3}

%\logo{logo-blue.png}
\cover{cover.jpg}

% 本文档命令
\usepackage{array}
\newcommand{\ccr}[1]{\makecell{{\color{#1}\rule{1cm}{1cm}}}}

\definecolor{customcolor}{RGB}{32,178,170}
\colorlet{coverlinecolor}{customcolor}

\begin{document}

\maketitle
\frontmatter


\tableofcontents

\mainmatter


\chapter{1月23日}
\section{碗內滾球的簡諧運動 $\star\ \star$}
參考自新概念Chap6. Ex1
\begin{example}
    質量為$m$,半徑為$r$的小球在半徑為$R$的半球形固定大碗內作純滾動($2r<R$),假設$\theta$為球心和碗心連線
    與鉛直線的夾角,$\phi$為滾動時對於球心的角位移,並設重力加速度為$g$(實心小球對於球心的轉動慣量為$\frac{2}{5}mr^2$)
    \begin{enumerate}
        \item 求$\dot{\theta}$和$\dot{\phi}$的關係 
        \item 求位能的表示式
        \item 求動能的表示式
        \item 此運動的運動方程(Hint:利用力學能守恆$dE/dt=0$)
        \item 在小幅度振盪的情況,此運動為簡諧運動,求SHM的週期
    \end{enumerate}
\end{example}
\newpage

\section{Larmor Formula}
\begin{example}
    根據Maxwell電動力學中,說明帶電粒子有加速度,即會輻射電磁波。而在1897年,物理學家Joseph Larmor,提出著名的Larmor公式
    \begin{equation}
        P = \frac{e^2 a^2}{6\pi \epsilon_{0} c^3}
    \end{equation}
    此公式描述了在非相對論情況下,若粒子帶有加速度$a$,將會輻射出功率為$P$的電磁波。
    結合半古典(Semi-classical)的波耳氫原子模型
    (Bohr Hydrogen Atomic Model),假設電子的質量為m,庫倫常數以$\frac{1}{4\pi\epsilon_0}$表示,回答以下問題。
    \begin{enumerate}
        \item 證明在我們的模型中,大多數的情況為非相對情況下,即$v<<c$。
        \item 計算Bohr model的壽命(假設運動一週的軌道都可以近似成圓,且電子在基態軌道,波耳半徑為0.53$\mbox{\normalfont\AA}$)。
    \end{enumerate}
    Ref. Griffiths Prob11.14 \\
    Ref. https://www.physics.princeton.edu/~mcdonald/examples/orbitdecay.pdf
\end{example}







\end{document}