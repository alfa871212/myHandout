\documentclass[cn,10pt,math=newtx,chinesefont=founder]{elegantbook}

\title{波動與光題目精選}
\subtitle{2021年}

\author{李宥頡}
\institute{National Taiwan University}
%\date{May 2, 2021}
%\version{4.1}
%\bioinfo{自定义}{信息}

%\extrainfo{寶寶肚子餓—— 顏利蓁}

\setcounter{tocdepth}{3}

%\logo{logo-blue.png}
\cover{cover.jpg}

% 本文档命令
\usepackage{array}
\newcommand{\ccr}[1]{\makecell{{\color{#1}\rule{1cm}{1cm}}}}

\definecolor{customcolor}{RGB}{32,178,170}
\colorlet{coverlinecolor}{customcolor}

\begin{document}

\maketitle
\frontmatter

\chapter*{序}
\tableofcontents

\mainmatter

\chapter{波動}
\begin{example}
    如圖,有一脈動由左向右行進,則在此一瞬時,介質中波形所在處之各點振動的瞬時速度和
    位置之關係為何?(縱軸表速度,橫軸表位置)\\
    \rightline{[雄中科學班108]}

\end{example}
\begin{solution}

\end{solution}
\begin{figure}[htbp]
    \includegraphics[width=\textwidth]{image/雄108_2.png}
\end{figure}
\newpage

\begin{example}
    有一個等速向右進行的週期波,時間為零時,其圖形如圖(一),經$\frac{1}{400}$秒後,其波形如圖
    (二),求此波的頻率最小為何?\\
    \rightline{[雄中科學班105]}
\end{example}
\begin{solution}
    
\end{solution}
\begin{figure}[htbp]
    \flushright
    \includegraphics[width=0.5\textwidth]{image/雄105_2.png}
\end{figure}
\newpage

\begin{example}
    圖中的實線為一列向右方行進的週期波,在$t=0$秒時的部分
    波形,而虛線則為此週期波在$t =0.5$秒時的部分波形。若此
    列橫波之週期為T,且 0.3秒<T<1.2秒,則此橫波的波速為
    何?\\
    \rightline{[雄中科學班104]}
\end{example}
\begin{solution}
    
\end{solution}
\begin{figure}[htbp]
    \flushright
    \includegraphics[width=0.3\textwidth]{image/雄104_1.png}
\end{figure}



\chapter{光學}

\begin{example}
    如圖所示,兩相同的玻璃直角三稜鏡 ABC,兩者的 AC 面平行放置,其間是透明空氣(可視為真空)。一單色
    細光束 O 垂直於 AB 面入射,則在圖示的出射光線中可能為何者?(全對才給分)\\
    \rightline{[雄中科學班108]}
\end{example}
\begin{solution}
    
\end{solution}
\begin{figure}[htbp]
    \flushright
    \includegraphics[width=0.3\textwidth]{image/雄108_1.png}
\end{figure}
\newpage

\begin{example}
    如圖所示,兩平行的平面鏡 A 與 B 相距 24 cm,一點光源 S 位在 A 鏡的前方 8.0 cm 處,
    則在 A 鏡後會形成無窮多個虛像,令距離 A 鏡最近的虛像為$P_1$,距離 A 鏡第二近的虛像為$P_2$………。
    則當點光源 S 以$4 cm/s$的速度在 Z 軸上向右運動的瞬間,$P_2$的速度為何?(量值及方向全對才給分)
    \rightline{雄中科學班108}
\end{example}
\begin{solution}
    
\end{solution}
\begin{figure}[htbp]
    \flushright
    \includegraphics[width=0.3\textwidth]{image/雄108_3.png}
\end{figure}
\newpage

\begin{example}
    某人位於一直立的平面鏡前 2 m 處,見一光點在平面鏡前 5 m 處之牆面,沿與
    平面鏡鏡寬平行方向,以 3$m/s$作等速運動,若平面鏡鏡寬 1 m,則人於平面鏡
    中能見到光點虛像的時間為多少秒?\\
    \rightline{[雄中科學班107]}
\end{example}
\begin{solution}
    
\end{solution}
\begin{figure}[htbp]
    \flushright
    \includegraphics[width=0.3\textwidth]{image/雄107_1.png}
\end{figure}
\newpage

\begin{example}
    如圖所示,若物體$AB$經由某透鏡折射後,成像為$A^{'} B^{'}$,則此透鏡為何種透鏡,
    位於圖中的哪區?\\
    \rightline{[雄中科學班105]}
\end{example}
\begin{solution}
    
\end{solution}
\begin{figure}[htbp]
    \flushright
    \includegraphics[width=0.3\textwidth]{image/雄105_1.png}
\end{figure}
\newpage

\begin{example}
    兄弟同處試衣間試衣,兄身高 180 cm,眼距頭頂 10 cm;弟身高 160 cm,眼距頭頂
    8 cm。欲在牆上固定一直立平面鏡使兩人皆能見到全身之像,則店家提供的平面鏡
    長度至少應為何?\\
    \rightline{[雄中科學班104]}
\end{example}
\begin{solution}
    
\end{solution}

\end{document}